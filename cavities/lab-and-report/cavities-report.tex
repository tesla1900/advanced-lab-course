\documentclass[a4paper]{article}

\usepackage[utf8]{inputenc}
\usepackage[T1]{fontenc}
\usepackage{textcomp}
\usepackage[english]{babel}
\usepackage{amsmath, amssymb}


% figure support
\usepackage{import}
\usepackage{xifthen}
\pdfminorversion=7
\usepackage{pdfpages}
\usepackage{transparent}
\usepackage{physics}
\setlength{\parindent}{0pt}
\newcommand{\incfig}[1]{%
		\def\svgwidth{\columnwidth}
		\import{./figures/}{#1.pdf_tex}

}

\pdfsuppresswarningpagegroup=1

\begin{document}
\section{Introduction}



\section{Theory}


\section{Waveguides}
In order to understand cavities, we start off with the discussion on waveguides. An electromagnetic wave, when confined to the interior of a hollow pipe is called a waveguide. We shall closely follow the derivation from \cite{}.The generalized Maxwell equations in terms of E-field and H-field are given by:

\begin{align}
		\curl {\vb{{E}}} = - \pdv{{B}}{t} \\
		\div{\vb{D}} = \rho \\
		\curl {\vb{{H}}} = \vec{j} + \pdv{{D}}{t} \\
		\div{\vb{{B}}} = 0 
\end{align}

In vacuum, these equations become: 
\begin{align}
		\curl {\vb{{E}}} = \mu_0 \cdot \pdv{{H}}{t} \\ 
		\div{\vb{{E}}} = 0 \\
		\curl {\vb{{H}}} = \epsilon_{0} \cdot \pdv{{E}}{t} \\
		\div{\vb{{H}}} = 0
\end{align}

Upon solving these equations, we get plane wave solutions which looks like this: 
\begin{align}
		\Delta \vec{E}\left(\vec{r}\right) + \frac{\omega}{c^2} \vec{E}\left(\vec{r}\right) = 0 \\
		\Delta \vec{B}\left(\vec{r}\right) + \frac{\omega}{c^2} \vec{B}\left(\vec{r}\right) = 0 
\end{align}

Now, let us look at a waveguide which is aligned along the z-direction. Taking the ansatz $\vec{E} = \vec{E}(x,y).e^{i(\omega t -kz)}$ and the separation $\Delta = \Delta_{\perp} + \pdv[2]{}{z}$, for the longitudinal fields yields: 

\begin{align}
		\Delta_{\perp}E_{z} + k_{c}^2 E_{z} = 0 \\
		\Delta_{\perp}H_{z} + k_{c}^2 H_{z} = 0 
\end{align}
where 
\[
		k_{c}^2 = \frac{\omega^2 }{c^2 } - k^2   
\]
The quantity $k_{c} $ is called the critical wave number and is a characteristic property of the cavity, as we shall see. 
Solving these equations further, we see that it is sufficient to know the longitudinal fields, $E_{z} $ and $B_{z} $ because we can easily determine the transverse components from them. 
Now, there are two waves to classify these waves: 
\begin{enumerate}
		\item $k_{c}^2 = 0  $ 
				\begin{enumerate}
						\item $\vec{\Delta_{\perp} }E_{z}  \ne 0$ and $\vec{\Delta}_{\perp}H_{z} \ne 0  $: HE or EH hybrid waves.
						\item $\vec{\Delta_{\perp} }E_{z}  = 0$ and $\vec{\Delta}_{\perp}H_{z} = 0  $: TEM waves.
				\end{enumerate}
		\item $k_{c}^2 \ne 0  $ 
		In this case we do not get any propagation if $\omega \leq c \cdot k_{c} $. These waves are called evanescent waves or the cut off condition. The possible propagation modes are: 
		\begin{enumerate}
				\item $E_{z}=0 $: TE (transversal electric) or H waves.
				\item $H_{z}=0 $: TM (transversal magnetic) or E waves.
		\end{enumerate}
		
\end{enumerate}

Corresponding to this critical wave, we have a critical frequency, which is $\omega_{c} = k_{c}.c $, below which there is no propagation. 
One interesting thing to note here is that for a hollow waveguide, only TE or TM modes are possible, TEM is not, because no wave would exist in this case \cite{}. But for a coaxial cable, which consists of straight wire surrounded by a conduction sheath, we can get TEM modes.

\subsection{Cylindrical waveguides}


\end{document}
