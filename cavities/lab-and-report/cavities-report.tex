\documentclass[a4paper]{article}

% \usepackage[utf8]{inputenc}
% \usepackage[T1]{fontenc}
% \usepackage{textcomp}
\usepackage[english]{babel}
\usepackage{amsmath, amssymb}
\usepackage[separate-uncertainty=true, multi-part-units=single]{siunitx}
\usepackage[]{subfig}
\usepackage[colorlinks=true, anchorcolor=blue, linkcolor=blue, citecolor=blue, bookmarks=false,hyperfootnotes=false]{hyperref}
\usepackage[margin=1in]{geometry}
\usepackage{color,soul}

% figure support
\usepackage{import}
\usepackage{xifthen}
\pdfminorversion=7
\usepackage{pdfpages}
\usepackage{transparent}
\usepackage{physics}
\graphicspath{ {./figures/} }
\setlength{\parindent}{0pt}
\usepackage{chngcntr}
\numberwithin{equation}{section}
\counterwithin{figure}{section}
\newcommand{\incfig}[1]{%
		\def\svgwidth{\columnwidth}
		\import{./figures/}{#1.pdf_tex}

}

\pdfsuppresswarningpagegroup=1

% for citations / references
\usepackage[style=ieee]{biblatex}
\addbibresource{cavities-report.bib}

\begin{document}
\section{Introduction}



\section{Theory}


\section{Waveguides}
In order to understand cavities, we start off with the discussion on waveguides.
An electromagnetic wave, when confined to the interior of a hollow pipe is
called a waveguide. We shall closely follow the derivation from \cite{Switka22}.The
generalized Maxwell equations in terms of E-field and H-field are given by:

\begin{align}
		\curl {\vb{{E}}} &= - \pdv{{B}}{t} \\
		\div{\vb{D}} &= \rho \\
		\curl {\vb{{H}}} &= \vec{j} + \pdv{{D}}{t} \\
		\div{\vb{{B}}} &= 0 
\end{align}

In vacuum, these equations become: 
\begin{align}
		\curl {\vb{{E}}} &= \mu_0 \cdot \pdv{{H}}{t} \\ 
		\div{\vb{{E}}} &= 0 \\
		\curl {\vb{{H}}} &= \epsilon_{0} \cdot \pdv{{E}}{t} \\
		\div{\vb{{H}}} &= 0
\end{align}

Upon solving these equations, we get plane wave solutions which looks like this: 
\begin{align*}
		\Delta \vec{E}\left(\vec{r}\right) + \frac{\omega}{c^2} \vec{E}\left(\vec{r}\right) = 0 \\
		\Delta \vec{B}\left(\vec{r}\right) + \frac{\omega}{c^2} \vec{B}\left(\vec{r}\right) = 0 
\end{align*}

Now, let us look at a waveguide which is aligned along the z-direction. Taking
the ansatz $\vec{E} = \vec{E}(x,y).e^{i(\omega t -kz)}$ and the separation
$\Delta = \Delta_{\perp} + \pdv[2]{}{z}$, for the longitudinal fields yields: 

\begin{align}
		\Delta_{\perp}E_{z} + k_{c}^2 E_{z} = 0 \\
		\Delta_{\perp}H_{z} + k_{c}^2 H_{z} = 0 
\end{align}
where 
\[
		k_{c}^2 = \frac{\omega^2 }{c^2 } - k^2   
\]
The quantity $k_{c} $ is called the critical wave number and is a characteristic
property of the cavity, as we shall see. Solving these equations further, we see
that it is sufficient to know the longitudinal fields, $E_{z} $ and $B_{z} $
because we can easily determine the transverse components from them. 
\begin{align} 
		ik_{c}^2 \vec{E}_{\perp} = k \vec{\nabla}_{\perp}E_{z} + \omega \mu _{0} \vec{\nabla}_{\perp}H_{z} \times \hat{e}_{z} \label{trans1} \\
		ik_{c}^2 \vec{H}_{\perp} = k \vec{\nabla}_{\perp}H_{z} - \omega \epsilon_{0} \vec{\nabla}_{\perp} E_{z} \times \hat{e}_{z} \label{trans2}     
\end{align}

Now, there are two waves to classify these waves: 
\begin{enumerate}
		\item $k_{c}^2 = 0  $ 
				\begin{enumerate}
						\item $\vec{\Delta_{\perp} }E_{z}  \ne 0$ and
						$\vec{\Delta}_{\perp}H_{z} \ne 0  $: HE or EH hybrid
						waves.
						\item $\vec{\Delta_{\perp} }E_{z}  = 0$ and
						$\vec{\Delta}_{\perp}H_{z} = 0  $: TEM waves.
				\end{enumerate}
		\item $k_{c}^2 \ne 0  $ In this case we do not get any propagation if
		$\omega \leq c \cdot k_{c} $. These waves are called evanescent waves or
		the cut off condition. The possible propagation modes are: 
		\begin{enumerate}
				\item $E_{z}=0 $: TE (transversal electric) or H waves.
				\item $H_{z}=0 $: TM (transversal magnetic) or E waves.
		\end{enumerate}
		
\end{enumerate}

Corresponding to this critical wave, we have a critical frequency, which is
$\omega_{c}=k_{c}.c $, below which there is no propagation. One interesting
thing to note here is that for a hollow waveguide, only TE or TM modes are
possible, TEM is not, because no wave would exist in this case \cite{Griffiths}. But for
a coaxial cable, which consists of straight wire surrounded by a conduction
sheath, we can get TEM modes.

\subsection{Cylindrical waveguides}
We now consider a cylindrical waveguide with an inner radius a. This imposes the
following boundary conditions to the walls of the waveguide:
\begin{itemize}
		\item $E_{\phi} = 0$, $E_{z} = 0 $ \text{for} $r = a$ 
		\item $H_{r} = 0 $ \text{for} $r = a$ 
\end{itemize}

The field distribution solution for a cylindrical waveguide can be separated
into angular and radial parts, whose solutions are then given by the
Bessel/Neumann functions. These can be then substituted in the
eq.(\ref{trans1})-(\ref{trans2}) and upon imposing the constraints from the
boundary condition gives us the corresponding TE- and TM-modes.

\subsection{Cylindrical waveguides resonator}
The time has now come, to talk about cavities itself. If we now insert two
conducting plates perpendicular to the z-axis, the incoming wave is reflected
completely, giving us standing waves. Because of this, the z-dependence changes
like: 

\[
		a \cdot e^{ikz} \implies A \cdot \sin\left(kz + \phi_{0}\right)    
\]
The following condition is imposed so that the longitudinal boundary conditions
are fulfilled: $k = p\cdot \pi /l $. The longitudinal field looks like: 
\begin{align*}
		\mathbf{TE_{mnp}-Modes}&: H_{z} = H_{mn}\cdot J_{m} \left(k_{c}r \right) \cos\left(m \phi \right) \cdot \sin\left(p \pi / l \cdot z \right) \cdot e^{\omega_{mnp}t}; \\  &\text{ where } k_{c}a=j_{mn}^{'}  \\
		\mathbf{TM_{mnp}-Modes}&: E_{z} = E_{mn}\cdot J_{m}^{'}  \left(k_{c}r \right) \cos\left(m \phi \right) \cdot \cos\left(p \pi / l \cdot z \right) \cdot e^{\omega_{mnp}t}; \\  &\text{ where } k_{c}a=j_{mn} 
\end{align*}
For resonant frequency, we have:

\[
		\omega_{mnp} = c \cdot \sqrt{\left(j_{mn}/a\right)^2 + \left(p \pi / l\right)^2} 
\]

Here, $J_{m}$ is the $m$-th Bessel function, $J_{m}^{'} $ is its derivative. And
$j_{mn}$ and $j_{mn}^{'}$ are the $n$-th zeropoints of the $m$-th Bessel
function and its derivative.

The resonant modes can be written in the linear form as:

\begin{equation} \label{eqn:res_freq}
		\left(d \nu \right)^2 = \left(\frac{cj_{mn}^{(')}}{\pi}\right)^2 + \left(\frac{c}{2}\right)^2 p^2 \left(\frac{d}{l}\right)^2
\end{equation}

here $d$ is the diameter of the cavity. When we plot different modes on a graph,
we get a mode map (Fig.\ref{fig:mode}). The mode map allows one to read off the
resonant frequencies for different diameters and length of the cavity. 

\begin{figure}[hbt!]
    \centering
    \includegraphics[width=0.8\textwidth]{mode_map}
	\caption{Mode map for $p \leq 2$. \cite{Switka22}}
    \label{fig:mode}
\end{figure}

Since the derivative of the zeroth order Bessel function and the first order
Bessel function, $j^{'}_{0n}$ and $j_{1n}$, are equal, the corresponding TE- and
TM-modes have the same resonant frequencies. That is, $TE_{0np}$-modes and
$TM_{1np}$-modes have the same resonant frequency. 

\section{Oscillating circuit}
A cavity has a lot of characteristic quantities, which can be described by an
equivalent circuit (Fig.\ref{fig:circuit}). 
\begin{figure}[hbt!]
    \centering
    \includegraphics[width=0.8\textwidth]{circuit}
	\caption{Equivalent circuit of a cavity with loop coupling. \cite{Switka22}}
    \label{fig:circuit}
\end{figure}	

Coupling is a process through which electromagnetic waves can be coupled to a
waveguide, or in this case, to a cavity. There are several ways to couple and in
this experiment, we use loop coupling, which enables coupling to the magnetic
field inside the cavity. In the figure \ref{fig:circuit}, the LCR-circuit
represents the cavity. The step-down transformer represents the loop coupling,
$Z_{0}$ is the characteristic impedance and $R_{s}$ is the Shunt impedance. 

The characteristic quantities associated with the cavity are:
\begin{itemize}
		\item Quality factor 
		\item Coupling coefficient 
		\item Reflection coefficient 
		\item Shunt impedance
\end{itemize}

Let us look at these quantities in a bit more detail. 

\subsection{Quality factor}
Quality factor is a dimensionless quantity which describes how underdamped an
oscillator or resonator is. It is defined as: 
\begin{equation}
		Q_{0} = \frac{2 \cdot \pi \text{stored energy} }{\text{losses per period}} = \frac{2 \pi \cdot W }{T\cdot P} =  \frac{\omega_{0}\cdot W}{P}
\end{equation}
where $\omega_{0}$ is the angular resonant frequency. Looking at the case of
driven oscillations, the unloaded quality factor can be determined from the Full
Width Half Maximum (FWHM), $\Delta \omega_{H}$

\begin{equation} \label{eqn:fwhm}
		Q_{0} = \frac{\omega_{0}}{\Delta \omega_{H}}
\end{equation}

\subsection{Coupling coefficient}
The coupling coefficient is defined as:
\begin{equation} \label{eqn:coupling}
		\kappa = \frac{Z_{a}}{Z_{0}} = \frac{R_{s}}{n^2Z_{0}} = \frac{Q_{0}}{Q_{ext}}	
\end{equation}
where $n$ is the transformer turn ratio, $Q_{0}$ is the unloaded quality factor
and $Q_{ext}$ is the external quality factor.  
If we know the coupling coefficient, the unloaded quality factor can be
calculated using: 

\[
		\frac{1}{Q} = \frac{1}{Q_{0}} + \frac{1}{Q_{ext}}
\]

\begin{equation} \label{eqn:quality}
		Q_{0} = \left(1 + \kappa\right)\cdot Q
\end{equation}
\\
We also get 3 cases for the coupling coefficient, which are: 
\begin{itemize}
		\item $\kappa < 1$: undercritical coupling, $Q>Q_{0}/2$
		\item $\kappa = 1$: critical coupling, $Q = Q_{0}/2$ (no reflection)
		\item $\kappa > 1$: overcritical coupling, $Q<Q_{0}/2$ 
\end{itemize}

\subsection{Reflection coefficient}
In the conductor, we have an incoming wave ($\hat{U}_{+}, \hat{I}_{+}$) and
reflected wave ($\hat{U}_{-}, \hat{I}_{-}$). Hence we define the complex
reflection coefficient as:

\begin{equation} \label{eqn:refleccoeff}
		\rho = \frac{\hat{U}_{-}}{\hat{U}_{+}}
\end{equation}

The coupling coefficient and the reflection coefficient are related at resonance
by: 
\\
\begin{equation}
		\kappa = \left|\frac{1 + \rho}{1 - \rho} \right| 
\end{equation}
\\
We shall discuss more about this in the next section.

\subsection{Shunt impedance}
Shunt impedance tells us how much energy is gained by a charged particle when it
crosses the cavity. It is defined by: 

\begin{equation} \label{eqn:shunt}
		R_{s} = \frac{U^2}{2 P_{V}} = \frac{1}{P_{V}} \left|\int\limits_{L/2}^{-L/2} E_{0}\left(s \right) \cdot e^{i \omega_{0}s/c}\cdot ds \right|^2 
\end{equation}
\\
The impedance of the resonator in fig.\ref{fig:circuit} is a complex quantity. This value only becomes real at resonance and when it does, it is called the Shunt impedance. This value is typically of the order of $10^{6} \  \Omega$. This is the reason we use a step down transformer in the circuit, using loop coupling, as seen in Eq.(\ref{eqn:coupling}). 

\section{Scalar measurement of reflection coefficient}
The reflection coefficient we defined in the previous section (eqn. \ref{eqn:refleccoeff}) is a complex value. It looks like:

\[
		\rho(\Delta \omega) = \rho_{0} (\Delta \omega) \cdot e^{-2ikl} = \frac{\kappa - \left(1 + 2iQ_{0}\frac{\Delta \omega}{\omega}\right)}{\kappa + \left(1 + 2iQ_{0}\frac{\Delta \omega}{\omega}\right)} \cdot e^{-2ikl}
\]
\\
We get this when we investigate only take $\Delta \omega \ll \omega_{0}$. When this reflection coefficient is measured some distance $l$ away from the cavity, we get the delay factor of $e^{-2ikl}$.
\\ \\
Now, we can separate the real and imaginary part and get its modulus: 

\begin{equation} \label{eqn:scalar}
			\left| \rho (\Delta \omega) \right| = \left| \rho_{0}(\Delta \omega) \right| = \sqrt{\frac{(\kappa - 1)^2 + 4Q_{0}^2 \left(\Delta \omega / \omega \right)^2}{(\kappa + 1)^2 + 4Q_{0}^2 \left(\Delta \omega / \omega \right)^2}} 
\end{equation}
\\
The scalar network analyzer should then look like fig. \ref{fig:scalar}.

\begin{figure}[hbt!]
    \centering
    \includegraphics[width=0.8\textwidth]{scalar}
	\caption{Reflection coefficient for different values of coupling coefficient and Quality factors along with FWHM ($\Delta \omega_{H}$). \cite{Switka22}}
    \label{fig:scalar}
\end{figure}

From the figure, we can see that the reflection is minimum at resonance. At resonance, the value of the reflection coefficient is given by:

\begin{equation}
		\left| \rho (\Delta \omega = 0) \right| = \left| \frac{\kappa - 1}{\kappa +1} \right| 
\end{equation}

The equation can be inverted to calculate the value of coupling coefficient as
such:

\begin{equation}
	\kappa = 
	\begin{cases}
		\frac{1 + |\rho|}{1 - |\rho|} :& \rho > 0\\
		\frac{1 - |\rho|}{1 + |\rho|} :& \rho < 0
	\end{cases}
	\label{eqn:kappa_rho}
\end{equation}

As such, one cannot distinguish between $\rho >0$ and $\rho<0$. 

In order to calculate the value of reflection coefficient at FWHM, we use the following relation (which one can derive by using eqn.(\ref{eqn:fwhm}), (\ref{eqn:quality}) and (\ref{eqn:scalar}):

\begin{equation}
		\left| \rho (\Delta \omega_{H}/2) \right| = \frac{\sqrt{\kappa^2 + 1} }{\kappa +1} 
		\label{eqn:rho_fwhm}
\end{equation}

It is important to note that only the case of $\kappa = 1$ do we get the FWHM at $\rho = 1 / \sqrt{2} $. This is because of the way we define dB-values, which will be discussed in another section. \\
In all other cases, 

\[
		\left| \rho (\Delta \omega_{H}/2) \right| = \frac{\sqrt{\kappa^2 + 1} }{\kappa +1} \ne \frac{1}{\sqrt{2} } 
\]

The (loaded) quality factor $Q$ is then determined by following Eq.
\ref{eqn:fwhm}:
\begin{equation}
	Q = \frac{\omega_0}{\Delta\omega_H}
	\label{eqn:Q_scalar}
\end{equation}

The unloaded quality factor can easily be determined from Eq. \ref{eqn:Q_scalar}
above:
\begin{equation}
	Q_0 = (1 + \kappa)Q
	\label{eqn:Q0_scalar}
\end{equation}

\section{Vectorial measurement of reflection coefficient}
The vectorial reflection coefficient is given by: 

\begin{equation}
		\rho (\Delta \omega) = \frac{(\kappa - 1)^2 - 4Q_{0}^2 \left(\Delta \omega / \omega \right)^2 - 4i \kappa Q_{0} \Delta\omega / \omega}{(\kappa + 1)^2 + 4Q_{0}^2 \left(\Delta \omega / \omega \right)^2}  \cdot e^{-2ikl}
\end{equation}
\\
Neglecting the delay coefficient term, $e^{-2ikl}$, we get an equation, which when plotted on the complex plane (close to the resonance), $\rho_{0}$ describes a circle of radius r around ($x_{0}, y_{0}$).  The radius r is given by: 

\[
		r = \frac{\kappa}{1+ \kappa}
\]

We find that all the positions of circles are independent of quality factor and depend only on the coupling coefficient. Another interesting thing we find is that all these circles go through (-1, 0) (as shown in fig. \ref{fig:circles}). 

\begin{figure}[hbt!]
    \centering
    \includegraphics[width=0.8\textwidth]{circles}
	\caption{Different position and radii for different values of $\kappa$. \cite{Switka22}}
    \label{fig:circles}
\end{figure}

Now taking the delay coefficient into account, we see the circles rotate around the origin (fig. \ref{fig:delay}). For large values of quality factor, the distortion in the shape of the circle is negligible, but it is quite noticeable in low values of quality factor. We will be neglecting these distortions in our measurements because we such calibrations can be done only on expensive equipments. \\
In fig. \ref{fig:delay}, we also notice that we have an outer circle which is the reflection circle. It represents complete reflection of frequencies far away from the resonance one. The smaller circle is called the resonance circle. 

\begin{figure}[hbt!]
    \centering
    \includegraphics[width=0.8\textwidth]{delay}
	\caption{The delay coefficient rotates the circle around the origin. We can also see the Resonance and Reflection circle. \cite{Switka22}}
    \label{fig:delay}
\end{figure}

\subsection{Determining resonant frequency and coupling}
In the Vectorial Network Analyzer (VNA), we see something similar to Fig. \ref{fig:delay}. In order to determine the reflection coefficient, we rotate the circles around the origin such that the point of intersection of the resonant circle and reflection circle is on (-1, 0). Since the impedance of the resonator is real at resonance, the curve has to cross the real axis. Therefore, the point at which the resonance circle meets the real axis is the resonance frequency $\omega_{0}$. \\
In order to determine the reflection coefficient, we simply look at the distance between the resonant point and the origin, $d$, and the radius of the reflection circle, $R$. The reflection coefficient is then defined as: 

\[
		\rho_{0} = d/ R
\]

The coupling coefficient is given by: 

\[
		\kappa = \frac{1 + \rho_{0}}{1 - \rho_{0}}
\]


\subsection{Determining quality factor}
Similar to the scalar measurement, we use

\[
		\frac{\Delta \omega_{H}/2}{\omega_{0}} \approx \frac{1 + \kappa}{2Q_{0}}	
\]

and we get 

\[
		\rho_{0}(\pm \Delta \omega_{H}/2) = -\frac{1}{\kappa + 1} \mp i\frac{\kappa}{\kappa + 1}
\]

Therefore, the frequency shift $\Delta \omega$ from the resonant frequency and hence the FWHM can be determined by the intersection of the resonance circle around (0, $\pm i/2$).
This is essentially the frequency that is $\pi / 2$ rotation away from the resonant frequency on the resonance circle. 

\section{Bead pull measurement}
In order to measure the electric and magnetic field inside the cavity, we cannot use an antenna because that would disturb the field distribution inside. Instead, we introduce a small bead inside the cavity, a small perturbation. This bead, which is usually a dielectric or a conducting object, causes a shift in the resonant frequency, which can be measured. It can also be fixed with a constant excitation with $\omega_{0}$ and the change in reflection coefficient can be observed. These two things can be used to determine the fields inside the cavity. 

\subsection{Resonant bead pull measurement}
The case in which we measure the shift in the resonant frequency is called resonant bead pull measurement. A small teflon bead ($\epsilon_{r}$ = 2.1 , $r$ = 1mm) is moved in small increments inside the cavity. For each increment, the we get a shift in the resonance frequency $\Delta \omega(z)$. The Electric field, $E_{0}(z)$ is given by:

\begin{equation} \label{eqn:E0_res}
		E_{0}(z) = \sqrt{2 \cdot \frac{W}{\alpha_{S}}\cdot \frac{\Delta \omega(z)}{\omega_{0}}}
\end{equation}
where $\alpha_{s}$ is the perturbing constant: 

\[
		\alpha_{S} = \frac{1}{2} \cdot (\epsilon - \epsilon_{0}) \cdot V_{S}
\]
$V_{S}$ is the volume of the bead. 

\subsection{Non-resonant bead pull measurement}
The case in which we measure the change in reflection coefficient without changing the excitation is called non-resonant bead pull measurement. The electric field is determined by: 

\begin{equation} \label{eqn:E0_nonres}
		E_{0}(z) = \sqrt{\frac{(1 + \kappa)^2}{2 \kappa Q_{0}}\cdot \frac{W}{\alpha_{S}}\cdot \left| \Delta \rho(z) \right| }
\end{equation}

\subsection{Shunt impedance}
For determining the shunt impedance, it is sufficient to know the electric field. The accelerating voltage $U$ is given by: 

\[
		U = \int\limits_{0}^{L} E_{0}(z) \cdot dz
\]
Using this and Eqn. \ref{eqn:shunt}, we can define a delay coefficient as: 

\begin{equation}
		\Lambda = \left| \frac{\int\limits_{-L/2}^{L/2} E_{0}(s) \cdot e^{i \frac{\omega_{0}s}{c}\cdot ds}}{\int\limits_{-L/2}^{L/2} E_{0}(s) \cdot ds} \right| 
\end{equation}
And finally, we have for resonant method:

\begin{equation}
		R_{s} = \Lambda \cdot \frac{2Q_{0}}{\omega_{0}^2 \cdot \alpha_{S}} \cdot \left| \int\limits_{-L/2}^{L/2} \sqrt{\Delta \omega (z)} \cdot dz \right|^2
\end{equation}

And for non-resonant method we have: 

\begin{equation}
		R_{s} = \Lambda \cdot \frac{(1 + \kappa)^2}{2 \omega_{0} \kappa \alpha_{S}} \cdot \left| \int\limits_{-L/2}^{L/2} \sqrt{\Delta \rho (z)} \cdot dz \right|^2
\end{equation}

\section{What is dB?}

\section{What is dBm?}

\section{Exercise before the experiment}


\section{Experimental Setup}

In our experiment, we used three cylindrical cavities, each with a length of
$L_{cav} = \SI{20}{\milli\metre}$ with an inner diameter of $d_{cav} =
\SI{78.5}{\milli\metre}$. The first and second cavity are completely sealed and
are free to move, but differ in their coupling position to the radio frequency.
The third cavity is open from the sides and is mounted on a rail, in which its
position can be freely adjusted and is displayed on an external device
digitally. A thread with a teflon bead of radius $r_{bead} = \SI{1}{\milli\metre}$ and
relative permittivity of $\epsilon_{r, bead} = 2.1$ was placed on the same
mount such that the bead can pass through the cavity. The cavities used are shown in Fig.
\ref{fig:cavities_equipment}.

\begin{figure*}[hbt!]
	\centering
	\subfloat[\centering
	\label{fig:top_cavity}]{{\includegraphics[width=0.3\columnwidth]{cavity_top.jpg}}}
	\quad
	\subfloat[\centering
	\label{fig:mounted_side_cavity}]{{\includegraphics[width=0.6\columnwidth]{cavity_side_rail.png}}}

	\caption{The three cavities used in this experiment. (a): The cavity with
	top coupling. The connector that couples the radio frequency can be
	observed. (b) The cavity with side coupling (right) and the mounted cavity
	(left) \cite*{Switka22}. The detector used to measure the position of the
	mounted cavity is also shown.}
	\label{fig:cavities_equipment}
\end{figure*}

To measure the resonant frequencies, reflection coefficients, and other relevant
quantities in our experiment, we utilize a Vector Network Analyzer (VNA) shown
in Fig. \ref{fig:VNA}. The VNA generates a signal from its AC source and it is
transmitted through the coaxial cable connected to the ports of the VNA. The
resulting transmission or reflection of the signal is then shown on the display
as a function of the frequency of the signal (in Hz). The units displayed can be
altered between $U$, the ratio between output and input signal, and $dB = 20
\log_{10}(U)$. \par  

\begin{figure*}[hbt!]
	\centering
	\includegraphics[width=0.5\columnwidth]{VNA.jpg}
	\caption{The Rhode \& Schwarz Vector Network Analyzer used in this experiment.}

	\label{fig:VNA}
\end{figure*}

\section{Procedure}

\subsection{Attenuation and Reflection in Coaxial Cables}

As the coaxial cable connects the cavity to the VNA, we need to choose a cable
such that it does not interfere with the measurement of the cavity's response.
To determine the optimal coaxial cable to use, we thus measured the attenuation
and reflection of different coaxial cables at different frequency ranges. In our
experiment, this was done for the RG-142 and ST-18 cables with a length of
$l_{RG142}=\SI{51.02 \pm 0.10}{\centi\meter}$ and $l_{ST18}=\SI{182.55 \pm
0.10}{\centi\meter}$ respectively. The uncertainty of the length was obtained
from the width of the scale of the measuring apparatus. \par 

To measure the attenuation and reflection, we first calibrated the VNA to the
silver RG402 coaxial cable with a length of $l_{RG402} = \SI{30.02 \pm
0.01}{\centi\meter}$. The calibration process was performed using a full-two
port calibration with the TOSM algorithm provided by the VNA. The open, short,
and load (with termination impedance of $\SI{50}{\ohm}$) at the two ports (port
2 on the VNA and the coaxial cable) were then calibrated by utilizing the
provided calibration kit (Fig. \ref{fig:calibration_kit}). 

\begin{figure*}[hbt!]
	\centering
	\includegraphics[width=0.6\columnwidth]{calibration_kit.jpg}
	\caption{The open (left), short (middle) and load (right) used for
			calibration in this experiment.}

	\label{fig:calibration_kit}
\end{figure*}

The attenuation of the signal was then measured by connecting the end of the
coaxial cable to the input port (port 2) on the VNA. We then connect the same
load from the calibration kit to the end of the coaxial cable to measure the
reflection coefficient. Both measurements were performed to verify that the VNA
was properly calibrated. \par 

After the calibration was performed with the silver cable, we connected the two
coaxial cables to the silver cable and measured their attenuation and reflection
coefficient by performing the same procedure as with the silver one. The raw
measurements obtained are shown in Fig.  \ref{fig:calib_raw}. The fluctuations in the signal were taken into account as the
uncertainty in the coefficients. We neglected effects due to the connector
between the two cables.

\begin{figure*}[hbt!]
	\centering
	\subfloat[Attenuation for the RG142 cable.
	]{{\includegraphics[width=0.48\columnwidth]{calib_brown_s21.png}}}
	\quad
	\subfloat[Attenuation for the ST18
	cable.]{{\includegraphics[width=0.48\columnwidth]{calib_blue_s21.png}}}
	\quad
	\subfloat[Reflection for the RG142
	cable.]{{\includegraphics[width=0.48\columnwidth]{calib_brown_s11.png}}}
	\quad
	\subfloat[Reflection for the ST18
	cable.]{{\includegraphics[width=0.48\columnwidth]{calib_blue_s11.png}}}

	\caption{The attenuation and reflection for the RG142 and ST18 coaxial
	cables. The values at several frequencies are also shown on the top right of
	each figure. }
	\label{fig:calib_raw}
\end{figure*}

The transmission and reflection at several frequencies were then tabulated and were used to determine the attenuation
of the cable. The attenuation $\lambda$ (in units of dB m$^{-1}$) of a coaxial cable can be determined by the ratio of the
reflected and transmitted power $P_R$, $P_T$ as such:
\begin{equation}
	\lambda = 20 \log_{10}\left(\frac{P_{in}}{P_{out}}\right) = 20 \log_{10}\left(\frac{P_R}{P_T}\right)
\end{equation} 

The resulting attenuation was then compared to those provided in the datasheets
from the manufacturers.\par 

\subsection{Scalar Measurement}

We then connected the brown coaxial cable to the top cavity (cavity with top
coupling), and the resulting reflection coefficient was displayed on the VNA.
We then pinpointed several points in frequency in which a sharp dip in the
reflection curve was observed. Such points corresponded to 
resonant modes in the cavity. Before taking the measurement of each resonant frequency, the VNA was calibrated
using a full one-port procedure, calibration with the kit from Fig.
\ref{fig:calibration_kit} as before. An example of a resonance in the
reflection curve is shown in Fig. \ref{fig:resonance_raw}. The corresponding
reflection coefficient and resonant frequency was then measured. The
fluctuations and width of the scales taken into account as uncertainties in the
reflection coefficient. The step size of the VNA was also considered as the
uncertainty in frequency measurements.
\par 

\begin{figure*}[hbt!]
	\centering
	\includegraphics[width=0.6\columnwidth]{resonant1.1.png}
	\caption{Reflection curve at a resonant frequency of $\omega_0 =
	\SI{2.992206875}{\giga\hertz}$. }

	\label{fig:resonance_raw}
\end{figure*}

To obtain the full-width half-maximum (FWHM) of the curve, the reflection
coefficient obtained at resonance $|\rho(\omega_0)|$ was used to determine the
coupling coefficient as shown in Eq. \ref{eqn:kappa_rho}. The reflection
coefficient at $\frac{\Delta\omega_H}{2}$ was then evaluated by using Eq.
\ref{eqn:rho_fwhm} and the corresponding frequency values were measured using
the display. The loaded and unloaded quality factor was then determined using Eq. \ref{eqn:Q_scalar} and
\ref{eqn:Q0_scalar} respectively. Gaussian error propagation techniques were
utilized to obtain the correct uncertainties for relevant quantities. \par 


This procedure was performed for each observed resonance peak until $\SI{8}{\giga\hertz}$, the
maximal frequency in which the VNA can measure. The same analysis was repeated
for the side cavity, and the resonant frequency obtained in both cases were
compared with the analytical resonant frequencies evaluated from Eq. \ref{eqn:res_freq}.

\subsection{Vectorial Measurement}

We then used the top cavity to determine the resonant frequency and FWHM for the
first fundamental eigenmode vectorially. This was done using the reflection
curves as shown in Fig. \ref{fig:delay}, which is done by switching the display on the VNA to the complex representation. \par 

The resonant frequency was first determined by using the same method as with the
scalar analysis. We then switched to the complex representation and rotated the
curve such that the point on the curve at resonance intersects with the real
axis. The FWHM was then determined by first evaluating for the coupling
coefficient, using the same equations as with the scalar analysis. The
corresponding values of $|\rho(\frac{\Delta\omega_H}{2})|$ were then determined
on the resonant circle, and the resulting frequency values were tabulated.
The fluctuations of the measurement were again taken into account as uncertainties
in our measurement. \par 

The obtained values were then used to determine the loaded, unloaded and external quality
factor $Q$, $Q_0$, $Q_{ext}$, as well as the power loss of the cavity. This was first
performed without calibrating the VNA, and the same procedure was repeated after
calibrating the VNA. The calibration was performed as with the scalar case. 
The results obtained from both cases were compared and discussed. A raw measurement of the resonant and reflection curves in
both uncalibrated and calibrated cases are shown in Fig.
\ref{fig:vectorial_raw}.   

\begin{figure*}[hbt!]
	\centering
	\subfloat[Uncalibrated]{{\includegraphics[width=0.7\columnwidth]{vectorial1.1.png}}}
	\quad
	\subfloat[Calibrated]{{\includegraphics[width=0.7\columnwidth]{vectorial-smith.png}}}

	\caption{Vectorial measurement of the fundamental resonant mode (a) without
	and (b) with calibration. The curves are represented in linear polar and
	Smith representation respectively. The resonant frequency and points at
	half of the FWHM are indicated with markers 1, 2, 3 respectively.
	}
	\label{fig:vectorial_raw}
\end{figure*}

\subsection{Bead-Pull Measurement}
 

We then connected the coaxial cable to the cavity mounted on the rail. Using the
same method as with the scalar case, the location of the fundamental resonant
mode was determined and the resonant frequency and FWHM was tabulated. These
values were used to evaluate the quality factor $Q$, stored energy $W$, and the
power loss of the cavity. \par 

The cavity was then positioned such that the teflon bead was outside of the
cavity. The position of the cavity was then adjusted, and using the scalar
representation the change in resonant frequency $\Delta\omega_0$ and change in
reflection coefficient at resonant frequency $|\Delta\rho(\omega_0)|$ were
tabulated. The fluctuations and stepsize of the measurements were taken into
account as the uncertainties in measurement. This was performed until the bead
was completely outside of the cavity, which corresponds to a length of
approximately $\SI{40}{\milli\metre}$. The raw data can be observed in the
Appendix section. \par 

Using Eq. \ref{eqn:E0_res} and \ref{eqn:E0_nonres}, the unperturbed electric
field strength $E_0(z)$ was then determined for each position of the bead. Using
the power loss obtained from above, we then evaluated the shunt impedance $R_s$ by
utilizing cubic interpolations and Gauss-Legendre quadrature integration
techniques. The transit time factor $\cos(\frac{\omega_0 z}{c})$ was also taken
into account when evaluating the shunt impedance, and the electric fields when
considering such time factor was compared to those without this factor. \par 

Finally, the energy gain of an accelerating particle was determined for several
input powers. This was used to observe if our cavity was suited for particle
accelerator experiments.



\section{Results}

\subsection{Attenuation of Coaxial Cables}


\subsection{Scalar Measurement}




\printbibliography

\end{document}
