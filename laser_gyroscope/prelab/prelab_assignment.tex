\documentclass[a4paper]{article}

\usepackage[utf8]{inputenc}
\usepackage[T1]{fontenc}
\usepackage{textcomp}
\usepackage[english]{babel}
\usepackage{amsmath, amssymb}
\usepackage[marginparwidth=1.75cm]{geometry}

% figure support
\usepackage{import}
\usepackage{xifthen}
\pdfminorversion=7
\usepackage{pdfpages}
\usepackage{transparent}
\usepackage{graphicx}
\newcommand{\incfig}[1]{%
		\def\svgwidth{\columnwidth}
		\import{./figures/}{#1.pdf_tex}
}

\pdfsuppresswarningpagegroup=1

\begin{document}
\title{A249: Laser Gyroscope}
\maketitle 

In here we will present the tasks that we have to complete before conducting the lab. 

\section{Getting Started with Gyroscopes}

We downloaded the \texttt{phyphox} app made by RWTH Aachen University, and we played around with the Gyroscope function. We rotated the phone in several directions 
to observe the relationship between the orientation of the phone and the corresponding coordinates used in the application. 



\end{document}